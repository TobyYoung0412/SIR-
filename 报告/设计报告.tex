\documentclass[supercite]{HustGraduPaper}
%进行个人信息设置
\title{多传感器融合技术}
\author{杨金昊} %作者姓名
\date{\today} %日期,默认当日
\school{自动化学院} %院系名称
\classnum{自卓1601} %专业班级
\stunum {U201614480} %学号
\instructor{张朴} %指导教师姓名

%添加自己要用的其他宏包
\usepackage{xltxtra}
\usepackage{bm}

\begin{document}
	%生成标题页 \maketitle[可选参数]
	%可选参数:
	%logo color=green/black 华中科技大学字样的颜色,绿色或者黑色,默认绿色
	%line length=12em 填写信息处横线的长度,默认12em
	%line font=huawenzhongsong 填写信息的字体,默认huawenzhongsong
	\maketitle[logo color= black]

	\clearpage %结束上一页
	\pagenumbering{Roman} %摘要页码为大写罗马数字

	\tableofcontents
	
	%生成目录 \tableofcontents[可选参数]
	%可选参数:
	%pagenum=yes/no/true/false 目录是否显示页码,默认为false
	%toc in toc=yes/no/true/false 目录中是否有目录及其页码,默认为false
	%level=4 目录级数,默认是4,即显示到subsubsubsection
	%section indent=0em 目录第一级的缩进,默认是0em
	%subsection indent=1.5em 目录第二级的缩进,默认是1.5em
	%subsubsection indent=3.8em 目录第三级的缩进,默认是3.8em
	%subsubsubsection indent=7em 目录第四级的缩进,默认是7em
	%paragraph indent=11em 目录第五级的缩进,默认是11em
	%subparagraph indent=13em 目录第六级的缩进,默认13em
	%indent=normal/noindent/hustnoindent/sameforsubandsubsub 快速缩进设置,具体见文档
	%dot sep=4.5 目录点间距,默认4.5
	%section dot sep=4.5 目录第一级的点间距,默认是4.5
	%subsection dot sep=4.5 目录第二级的点间距,默认是4.5
	%subsubsection dot sep=4.5 目录第三级的点间距,默认是4.5
	%subsubsubsection dot sep=4.5 目录第四级的点间距,默认是4.5
	%paragraph dot sep=4.5 目录第五级的点间距,默认是4.5
	%subparagraph dot sep=4.6 目录第六级的点间距,默认是4.5
	%请注意在合适的位置放置\pagenumbering{numstyle}使用新的页码
	\clearpage%结束上一页
	\pagenumbering{arabic} %正文页码为阿拉伯数字

	% 正文内容
	\section{试题建模}

	\subsection{试题描述}
		大多数传染病如天花、流感、肝炎、麻疹等治愈后均有很强的免疫力,所以病愈的人既非健康者(易感染者),也非病人(已感染者),他们已经退出传染系统。

	\subsection{模型假设}
	\begin{enumerate}
		\item H1N1流感传播期内,总人数为$N$不变,既不考虑生死,也不考虑迁移,人群分为易感染者$S$,发病人群$I$和退出人群 $R$(包括死亡者和治愈者)三类,时刻$r$内这三类人在总人数中所占比例分别为$s(t)$、$i(t)$、$r(t)$。
		\item 每个病人每天有效接触的平均人数是常数$\lambda$,称日接触率。当病人与健康者有效接触时,使健康者受感染变为病人。根据假设,每个病人每天可使$\lambda_s(t)$个健康者变为病人,因为病人数为$N_i(t)$.所以每天共有$\lambda N s(t) i(t)$个健康者被感染。
		\item 病人每天被治愈的占病人总数的比例为$\mu$,称为日治愈率,治愈的病人具有了免疫力,即治愈后不再会成为二次患者。
		\item $s(t)$、$i(t)$、$r(t)$之和是一个常数$1$。
	\end{enumerate}

	\subsection{建模过程}
	\subsubsection{初始模型}
	在这个初始模型中,假设时刻$t$的病人数$i(t)$是连续可谓的函数,并且每天每个病人有效接触的人数为常数$\lambda$,考察$t$到$t+\delta t$病人人数的增加就可得$$x(t+\delta t)-x(t) = \lambda x(t) \delta t$$
	在设$t = 0$时有$x_0$有个病人,即可得微分方程$$\frac{dx}{dt} = \lambda x, \quad x(0)=x_0$$


	\section{试题中实现的关键难点}

	\section{程序运行指南}

	\section{程序运行分析实例}


	\nocite{*}

	\bibliography{Bibs/mybib}
\end{document}
