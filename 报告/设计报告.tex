\documentclass[supercite]{HustGraduPaper}
%进行个人信息设置
\title{多传感器融合技术}
\author{杨金昊} %作者姓名
\date{\today} %日期,默认当日
\school{自动化学院} %院系名称
\classnum{自卓1601} %专业班级
\stunum {U201614480} %学号
\instructor{张朴} %指导教师姓名

%添加自己要用的其他宏包
\usepackage{xltxtra}
\usepackage{bm}

\begin{document}
	%生成标题页 \maketitle[可选参数]
	%可选参数:
	%logo color=green/black 华中科技大学字样的颜色,绿色或者黑色,默认绿色
	%line length=12em 填写信息处横线的长度,默认12em
	%line font=huawenzhongsong 填写信息的字体,默认huawenzhongsong
	\maketitle[logo color= black]

	\clearpage %结束上一页
	\pagenumbering{Roman} %摘要页码为大写罗马数字

	\tableofcontents
	
	%生成目录 \tableofcontents[可选参数]
	%可选参数:
	%pagenum=yes/no/true/false 目录是否显示页码,默认为false
	%toc in toc=yes/no/true/false 目录中是否有目录及其页码,默认为false
	%level=4 目录级数,默认是4,即显示到subsubsubsection
	%section indent=0em 目录第一级的缩进,默认是0em
	%subsection indent=1.5em 目录第二级的缩进,默认是1.5em
	%subsubsection indent=3.8em 目录第三级的缩进,默认是3.8em
	%subsubsubsection indent=7em 目录第四级的缩进,默认是7em
	%paragraph indent=11em 目录第五级的缩进,默认是11em
	%subparagraph indent=13em 目录第六级的缩进,默认13em
	%indent=normal/noindent/hustnoindent/sameforsubandsubsub 快速缩进设置,具体见文档
	%dot sep=4.5 目录点间距,默认4.5
	%section dot sep=4.5 目录第一级的点间距,默认是4.5
	%subsection dot sep=4.5 目录第二级的点间距,默认是4.5
	%subsubsection dot sep=4.5 目录第三级的点间距,默认是4.5
	%subsubsubsection dot sep=4.5 目录第四级的点间距,默认是4.5
	%paragraph dot sep=4.5 目录第五级的点间距,默认是4.5
	%subparagraph dot sep=4.6 目录第六级的点间距,默认是4.5
	%请注意在合适的位置放置\pagenumbering{numstyle}使用新的页码
	\clearpage%结束上一页
	\pagenumbering{arabic} %正文页码为阿拉伯数字

	% 正文内容
	\section{试题建模}

	\subsection{试题描述}
		大多数传染病如天花、流感、肝炎、麻疹等治愈后均有很强的免疫力,所以病愈的人既非健康者(易感染者),也非病人(已感染者),他们已经退出传染系统。

	\subsection{模型假设}
	\begin{enumerate}
		\item H1N1流感传播期内,总人数为$N$不变,既不考虑生死,也不考虑迁移,人群分为易感染者$S$,发病人群$I$和退出人群 $R$(包括死亡者和治愈者)三类,时刻$r$内这三类人在总人数中所占比例分别为$s(t)$、$i(t)$、$r(t)$。
		\item 每个病人每天有效接触的平均人数是常数$\lambda$,称日接触率。当病人与健康者有效接触时,使健康者受感染变为病人。根据假设,每个病人每天可使$\lambda_s(t)$个健康者变为病人,因为病人数为$N_i(t)$.所以每天共有$\lambda N s(t) i(t)$个健康者被感染。
		\item 病人每天被治愈的占病人总数的比例为$\mu$,称为日治愈率,治愈的病人具有了免疫力,即治愈后不再会成为二次患者。
		\item $s(t)$、$i(t)$、$r(t)$之和是一个常数$1$。
	\end{enumerate}

	\subsection{SIR建模过程}
	\subsubsection{初始模型}
	在这个初始模型中,假设时刻$t$的病人数$i(t)$是连续可谓的函数,并且每天每个病人有效接触的人数为常数$\lambda$,考察$t$到$t+\delta t$病人人数的增加就可得$$x(t+\delta t)-x(t) = \lambda x(t) \delta t$$
	在设$t = 0$时有$x_0$有个病人,即可得微分方程$$\frac{dx}{dt} = \lambda x, \quad x(0)=x_0$$
而该方程的解为$$x(t)=x_0e^{\lambda t}$$结果表明,随着t的增加,病人人数$x(t)$无限增长。

	\section{SI模型}
	在上一个模型中,在病人有效解除的人群中,有健康人也有病人,只有健康的人才可以被传染为病人,故在此模型中区分了这两种人。
	假设模型中的总人数$N$不变,人群也分为易感染者恶化易感染者两类,在时刻t这两类人在总人口中的比例分别记作$s(t)和i(t)$,每个人病人每天有效解除的平均人数为常数,成为日接触率,之后病人与健康者接触时,健康者才可以变为病人。
	在该模型中,每个病人每天可以使$\lambda s(t)$个健康者变为病人,因为病人总数为$Ni(t)$,所以每天总共有$\lambdaNs(t)i(t)$个健康者被感染。由此可得$$N\frac{di}{dt} = \lambda N s(i)t(i)$$  $$s(t)+i(t) = 1$$在记初时时刻$t=0$的时候病人比例为$i_0$。
	分析可得该模型为Logistic模型,当$t \to \inf$时$i \to 1$即所有人终将被传染,全部变为病人,这显然与事实仍有出入
	
	\section{SIR模型}
	假设总人数N不变,人群分为了健康者,病人和治愈免疫者三类,三类人在总数N中的比例分别计为$s(t),i(t),r(t)$病人的日接触率$\lambda$日治愈率$\mu$与SI模型相同。
	对于治愈免疫者而言有$$N\frac{dr}{dt}=\mu N i(t)$$
	所以SIR模型的方程可以写作
	$$\frac{di}{dt}=\lambda si-\mu i, \quad i(0)=i_0$$
	$$\frac{ds}{dt}=-\lambda si, \quad s(0)=s_0$$
	$$\frac{dr}{dt}=\mu i, \quad r(0)=r_0$$
	由于SIR模型无法求出解析解,我们采用数值计算和模拟仿真的方法求解

	\subsection{WS小世界模型}
	由于需要采用模拟仿真的方法仿真SIR模型,所以需要对现有的世界进行建模,世界是有人组成的,而连接人与人的就是人际关系。因此我以人际关系进行入手点进行模型构造,假设平均每个人认识K个人,根据六度分隔理论,我决定采用WS小世界模型。首先将每个人与理他最近的K个人进行连接,其次为了保证网络的边具有一定的随机性需要进行边的重新连接,最终生成的网络可以大致作为人口网络。
	
	\section{试题中实现的关键难点}
	\subsection{WS小世界模型的构建与显示}
	WS小世界模型需要直观的显示出来,最先采用的方案时将所有的点均匀放在一个大圆的圆弧上,但是随着点数的增加,圆弧上的点之间的间距见效已经无法分辨出不同点之间的区别,大量的边也会将圆形内部填满。
	\textbf{解决方案}
	在查阅了matlab官方的相关资料后,在2014以后的版本中,matlab对图的显示提供了一套完整的解决方案,只需要使用'graph(s,t)'其中s,t分别为起始节点矩阵和目标节点矩阵,计算好s,t矩阵后即可显示出美观的网络。
	
	\subsection{数值分析解}
	由于是第一次通过matlab进行数值分析的计算,因模型需要一次设置三个常微分方程在一次仿真中,开始时无从下手。
	\textbf{解决方案}
	查阅了相关资料,看了别人的一些事例后发现,只需要将三个常微分方程写到同一个矩阵中,在封装到一个函数里面,最后在进行数值分析计算的时候调用该函数即可解决问题。
	

	\subsection{进行模拟仿真}
	模拟仿真的时候需要通过生成随即概率,以进行病毒传染,病人治愈的模拟仿真,由于需要在人口网络上进行,怎么让SIR模型和人口网络同时知道该该节点时病人还是健康者带来了不小的困扰。
	\textbf{解决方案}
	由于在节点上直接标记的方法不是十分的便于执行,最后决定采用中间数组的方式,通过人口数组计算不同节点被传染的概率,以及每个病人的治愈概率。该数组时一个1*n维的数组,在代码中存在多个这样的数组,一方面是将人口网络压缩到了一维,方便了疾病传播过程中的仿真,另一方面,该数组时人口网络的索引,但又是独立于人口网络的,以至于传播过程中人口网络不受影响。

	\section{程序运行指南}

	\section{程序运行分析实例}


	\nocite{*}

	\bibliography{Bibs/mybib}
\end{document}
